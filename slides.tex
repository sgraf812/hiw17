\documentclass{haskellbeamer}

\title{Solving Data-flow Problems in Syntax Trees}
\date{September 9, 2017}

% Lattice

\newcommand{\llub}{\sqcup}
\newcommand{\lbiglub}{\bigsqcup}
\newcommand{\lleq}{\sqsubseteq}
\newcommand{\lless}{\sqsubset}
\newcommand{\lPair}[2]{\left\langle{}#1,#2\right\rangle{}}
\newcommand{\lTriple}[3]{\left\langle{}#1,#2,#3\right\rangle{}}
\newcommand{\both}{\,\&\,}
\newcommand{\mto}{\to_{+}}

% Domain

\newcommand{\sFVs}{\text{\textsf{FVs}}}
\newcommand{\sStr}{\text{\textsf{Str}}}
\newcommand{\sStrType}{\text{\textsf{StrType}}}
\newcommand{\sStrTrans}{\text{\textsf{StrTrans}}}

% Transfer function
\newcommand{\transfer}[2]{\mathcal{T}\llbracket#1\rrbracket_{#2}}
\newcommand*{\letupsc}{\textsc{LetUp}\xspace}
\newcommand*{\letdnsc}{\textsc{LetDn}\xspace}
\newcommand{\fix}{\keyword{fix}\xspace}
\newcommand{\up}{\sfop{up}}
\newcommand{\down}{\sfop{down}}

% Map Notation
\newcommand{\pfun}{\rightharpoonup}
\newcommand{\emptymap}{[]}
\newcommand{\id}[1]{#1}
\newcommand{\maplit}[3][\id]{\left[#1{#2\mapsto#3}\right]}
\newcommand{\restrict}[2]{#1\restriction_{#2}}


\begin{document}

\maketitle

\begin{frame}{Introduction}
  \begin{itemize}
    \item My master's thesis\footnote{\tiny\url{https://pp.ipd.kit.edu/uploads/publikationen/graf17masterarbeit.pdf}}: Call Arity vs. Demand Analysis
      \begin{itemize}
        \item Result: Usage Analysis generalising Call Arity
        \item Precision of Call Arity without co-call graphs
      \end{itemize}
    \item Requirements led to complex analysis order
    \item \emph{Specification} of data-flow problem decoupled from its \emph{solution}
  \end{itemize}
\end{frame}

\begin{frame}[fragile]{Strictness Analysis}
  \begin{itemize}
    \item Provides lower bounds on \emph{evaluation cardinality}
    \item Which variables are evaluated at least once?
      \begin{itemize}
        \item[$S$] Strict (Yes!)
        \item[$L$] Lazy (Not sure)
      \end{itemize}
    \item Enables call-by-value, unboxing
  \end{itemize}
  \begin{overprint}
    \onslide<1>
    \begin{center}
      \begin{minipage}{0.5\textwidth}
        \begin{haskell}
          main = do
            let  x = ... -- $S$
            let  y = ... -- $S$
            let  z = ... -- $L$
            print (x + if odd y then y else z)
        \end{haskell}
      \end{minipage}
    \end{center}
    \onslide<2>
    \begin{center}
      \begin{minipage}{0.5\textwidth}
        \begin{haskell}
          main = do
            let !x = ... -- $S$
            let !y = ... -- $S$
            let  z = ... -- $L$
            print (x + if odd y then y else z)
        \end{haskell}
      \end{minipage}
    \end{center}
  \end{overprint}
\end{frame}

\begin{frame}[fragile]{GHC's Demand Analyser}
  \begin{itemize}
    \item Performs strictness analysis (among other things)
    \item Fuels Worker/Wrapper transformation
    \item Backward analysis
      \begin{itemize}
        \item Which strictness does an expression place on its free variables?
        \item Which strictness does a function place its arguments?
      \end{itemize}
    \item \emph{Strictness type}: $\sStrType = \lPair{\sFVs \pfun \sStr}{\sStr^*}$
  \end{itemize}
\end{frame}

\begin{frame}[fragile]{Strictness Signatures}
  \begin{itemize}
    \item Looks at the right-hand side of \hs{const} before the \hs{let} body!
    \item \emph{Unleashes} strictness type of \hs{const}'s RHS at call sites
  \end{itemize}
  \begin{center}
    \begin{minipage}{0.8\textwidth}
      \begin{haskell}
        let const a b = a -- $\mathtt{const} :: \lPair{\emptymap}{[S, L]}$
        in const 
            y             -- $S$
            (fac 1000)    -- $L$
      \end{haskell}
    \end{minipage}
  \end{center}
\end{frame}

\begin{frame}[fragile]{Call Context Matters}
  \begin{overprint}
    \onslide<1-3>
    \begin{itemize}
      \item Whole expression is strict in \hs{z}
      \item Only digests \hs{f} for manifest arity 1, can't look under lambda
      \item \hs{f} is called with 2 arguments
      \item<4>[\ding{213}] Analyse bound function when incoming arity is known
    \end{itemize}
    \onslide<4->
    \begin{itemize}
      \item Solution: Analyse RHS when incoming arity is known
      \item Formally: Finite approximation of \emph{strictness transformer}
        \begin{itemize}
          \item $\sStrTrans = \mathbb{N} \to \sStrType$
        \end{itemize}
      \item Exploit laziness to memoise results?
    \end{itemize}
  \end{overprint}
  \begin{overprint}
    \onslide<1>
    \begin{center}
      \begin{minipage}{0.7\textwidth}
        \begin{haskell*}{escapeinside=||}
          let f x = -- $\mathtt{f} :: \lPair{\maplit{\mathtt{z}}{L}}{[S]}$
                if odd x
                  then \y -> y*|\color{red}\texttt{z}|
                  else \y -> y+|\color{red}\texttt{z}|
          in f 1 2
        \end{haskell*}
      \end{minipage}
    \end{center}
    \onslide<2>
    \begin{center}
      \begin{minipage}{0.7\textwidth}
        \begin{haskell*}{escapeinside=||}
          let f x = -- $\mathtt{f} :: \lPair{\maplit{\mathtt{z}}{L}}{[S]}$
                if odd x
                  then \y -> y*|\color{red}\texttt{z}|
                  else \y -> y+|\color{red}\texttt{z}|
          in seq (f 1) 42
        \end{haskell*}
      \end{minipage}
    \end{center}
    \onslide<3>
    \begin{center}
      \begin{minipage}{0.7\textwidth}
        \begin{haskell*}{escapeinside=||}
          let f x = -- $\mathtt{f} :: \lPair{\maplit{\mathtt{z}}{L}}{[S]}$
                if odd x
                  then \y -> y*z
                  else \y -> y+z
          in f 1 2
        \end{haskell*}
      \end{minipage}
    \end{center}
    \onslide<4>
    \begin{center}
      \begin{minipage}{0.7\textwidth}
        \begin{haskell*}{escapeinside=||}
          let f x = -- $\mathtt{f}_{\color{red}1} :: \lPair{\maplit{\mathtt{z}}{L}}{[S]}$
                if odd x
                  then \y -> y*z
                  else \y -> y+z
          in f 1 2
        \end{haskell*}
      \end{minipage}
    \end{center}
    \onslide<5>
    \begin{center}
      \begin{minipage}{0.7\textwidth}
        \begin{haskell*}{escapeinside=||}
          let f x = -- $\mathtt{f}_{\color{red}2} :: \lPair{\maplit{\mathtt{z}}{S}}{[S,S]}$
                if odd x
                  then \y -> y*z
                  else \y -> y+z
          in f 1 2
        \end{haskell*}
      \end{minipage}
    \end{center}
  \end{overprint}
\end{frame}
  
\begin{frame}[fragile]{Recursion}
  \begin{itemize}
    \item Exploit laziness to memoise approximations?
    \item[\xmark] Recursion leads to termination problems
    \item Rediscovered fixed-point iteration, detached from the syntax tree
    \item Leads to data-flow network, solved by worklist algorithm
  \end{itemize}
  \begin{center}
    \begin{minipage}{0.5\textwidth}
      \begin{haskell}
        let fac n = 
              if n == 0
                then 1
                else n * fac (n-1)
        in fac 12
      \end{haskell}
    \end{minipage}
  \end{center}
\end{frame}

\begin{frame}[fragile]{Recursive Example}
  \begin{itemize}
    \item Allocate nodes to break recursion
      \begin{itemize}
        \item One top-level node
        \item One node per pair of $(\text{\hs{let} binding}, \text{incoming arity})$
      \end{itemize}
    \item Initialise worklist to top-level node
    \item Initialise nodes with $\bot$
  \end{itemize}
  \begin{overprint}
    \begin{columns}
      \begin{column}{0.35\textwidth}
        \begin{haskell}
          let even 0 = True
              even n = odd (n-1)
              odd 0 = False
              odd n = even (n-1)
          in even 12
        \end{haskell}
      \end{column}
      \begin{column}{0.02\textwidth}
        {\Huge$\Rightarrow$}
      \end{column}
      \begin{column}{0.4\textwidth}
        \begin{tikzpicture}[->, thick]
          \useasboundingbox (-1,-2) rectangle (4,2);
          \tikzstyle{n}=[%
            fill=white,
            draw,
            circle,
            inner sep=0pt,
            minimum size=1.5cm
          ]
          \node[n] at (0,0) (root) 
            [label={[visible on=<4>]above:$\bot$}]
            {$\dfnode{<root>}{0}$};
          \foreach \i in {-0.05,0,0.05} {
            \node[n, visible on=<2->] at (3,1.5) (evenx) 
              [xshift=\i cm, yshift=-\i cm] {$\dfnode{even}{n}$};
            \node[n, visible on=<2->] at (3,-1.5) (oddx) 
              [xshift=\i cm, yshift=-\i cm] {$\dfnode{odd}{n}$};
          }
          \node[visible on=<4>] at ([shift={(90:1.1)}]evenx) {$\bot$};
          \node[visible on=<4>] at ([shift={(270:1.1)}]oddx) {$\bot$};
        \end{tikzpicture}
      \end{column}
    \end{columns}
    \onslide<3->
    \begin{columns}
      \begin{column}{0.55\textwidth}
        \hfill Worklist:
      \end{column}
      \begin{column}{0.55\textwidth}
        $\{\dfnode{<root>}{0}\}$ \hfill
      \end{column}
    \end{columns}
  \end{overprint}
\end{frame}

\begin{frame}[fragile]{Recursive Example}
  \begin{overprint}
    \begin{columns}
      \begin{column}{0.35\textwidth}
        \begin{haskell}
          let even 0 = True
              even n = odd (n-1)
              odd 0 = False
              odd n = even (n-1)
          in even 12
        \end{haskell}
      \end{column}
      \begin{column}{0.02\textwidth}
        {\Huge$\Rightarrow$}
      \end{column}
      \begin{column}{0.4\textwidth}
        \begin{tikzpicture}[->, thick]
          \tikzstyle{n}=[%
            fill=white,
            draw,
            circle,
            inner sep=0pt,
            minimum size=1.5cm
          ]
          \node[n] at (0,0) (root) [hl={<2-10>}, align=center] 
            [label={[visible on=<-9>]above:$\bot$},
             label={[visible on=<10->]above:$[]$}]
            {$\dfnode{<root>}{0}$};
          \foreach \i in {-0.05,0,0.05} {
            \node<-2>[n] at (3,1.5) (evenx) [xshift=\i cm, yshift=-\i cm] {$\dfnode{even}{n}$};
            \node<-3>[n] at (3,-1.5) (oddx) [xshift=\i cm, yshift=-\i cm] {$\dfnode{odd}{n}$};
          }
          \setcounter{beamerpauses}{3}
          \draw<3-> (root) edge [bend left, hl={<3,8>}] (even);
          \node[n, visible on=<3->, hl={<3-8,13>}] 
            at (3,1.5) (even) 
            [label={[visible on=<-5>]above:$\bot$},
             label={[visible on=<6-7>]above:$\bot=[S]$},
             label={[visible on=<8->]above:$[S]$}]
            {$\dfnode{even}{1}$};
          \draw<4-> (even) edge [bend left, hl={<4,7>}] (odd);
          \node[n, visible on=<4->, hl={<4-7,9,12-14>}] 
            at (3,-1.5) (odd) 
            [label={[visible on=<-6>]below:$\bot$},
             label={[visible on=<7->]below:$[S]$}] 
            {$\dfnode{odd}{1}$};
          \draw<5-> (odd) edge [bend left, hl={<5-6,13>}] (even);
          \draw<9-> (root) edge [bend right, dashed, hl={<9>}] (odd);
        \end{tikzpicture}
      \end{column}
    \end{columns}
    \begin{columns}
      \begin{column}{0.55\textwidth}
        \hfill Worklist:
      \end{column}
      \begin{column}{0.55\textwidth}
        \only<1>{$\{\alert{\dfnode{<root>}{0}}\}$}\only<2-5>{$\{\}$}\only<6-11>{$\{\alert<11>{\dfnode{odd}{1}}\}$}\only<12->{$\{\}$}
      \end{column}
    \end{columns}
  \end{overprint}
\end{frame}

\section{End}
\end{document}