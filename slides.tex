\documentclass{haskellbeamer}

\title{Solving Data-flow Problems in Syntax Trees}
\date{September 9, 2017}

\input{defs.tex}

\begin{document}

\maketitle

\begin{frame}{Introduction}
  \begin{itemize}
    \item My master's thesis\footnote{\tiny\url{https://pp.ipd.kit.edu/uploads/publikationen/graf17masterarbeit.pdf}}: Call Arity vs. Demand Analysis
      \begin{itemize}
        \item Result: Usage Analysis generalising Call Arity
        \item Precision of Call Arity without co-call graphs
      \end{itemize}
    \item Requirements led to complex analysis order
    \item \emph{Specification} of data-flow problem decoupled from its \emph{solution}
  \end{itemize}
\end{frame}

\begin{frame}[fragile]{Strictness Analysis}
  \begin{itemize}
    \item Provides lower bounds on \emph{evaluation cardinality}
    \item Which variables are evaluated at least once?
      \begin{itemize}
        \item[$S$] Strict (Yes!)
        \item[$L$] Lazy (Not sure)
      \end{itemize}
    \item Enables call-by-value, unboxing
  \end{itemize}
  \begin{overprint}
    \onslide<1>
    \begin{center}
      \begin{minipage}{0.5\textwidth}
        \begin{haskell}
          main = do
            let  x = ... -- $S$
            let  y = ... -- $S$
            let  z = ... -- $L$
            print (x + if odd y then y else z)
        \end{haskell}
      \end{minipage}
    \end{center}
    \onslide<2>
    \begin{center}
      \begin{minipage}{0.5\textwidth}
        \begin{haskell}
          main = do
            let !x = ... -- $S$
            let !y = ... -- $S$
            let  z = ... -- $L$
            print (x + if odd y then y else z)
        \end{haskell}
      \end{minipage}
    \end{center}
  \end{overprint}
\end{frame}

\begin{frame}[fragile]{GHC's Demand Analyser}
  \begin{itemize}
    \item Performs strictness analysis (among other things)
    \item Fuels Worker/Wrapper transformation
    \item Backward analysis
      \begin{itemize}
        \item Which strictness does an expression place on its free variables?
        \item Which strictness does a function place its arguments?
      \end{itemize}
  \end{itemize}
\end{frame}

\begin{frame}[fragile]{Strictness Signatures}
  \begin{itemize}
    \item Interprocedural data-flow
    \item Looks at the right-hand side of \hs{const} before the \hs{let} body!
    \item \emph{Strictness type}: $\sStrType = \lPair{\sFVs \pfun \sStr}{\sStrSig}$
    \item \emph{Unleashes} usage type of \hs{const}'s RHS at call sites
  \end{itemize}
  \begin{center}
    \begin{minipage}{0.8\textwidth}
      \begin{haskell}
        let const a b = a -- $\mathtt{const} :: \lPair{\emptymap}{S \to L \to \bullet}$
        in const 
            y             -- $S$
            (fac 1000)    -- $L$
      \end{haskell}
    \end{minipage}
  \end{center}
\end{frame}

\begin{frame}[fragile]{Call Context Matters}
\end{frame}
  
\section{End}
\end{document}